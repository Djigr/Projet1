\documentclass[]{article}
\usepackage{siunitx}
%opening
\title{Short project report : Protein Interaction Calculator}
\author{Diane DABIR MOGHADDAM}

\begin{document}

\maketitle


\section{Introduction}
PIC (Protein Interaction Calculator) is an open-source program using python3.7.

The program determines interactions within a protein, based on the informations contained in a PDB file of this protein.
Implemented interactions are : disulphide bonds, hydrophobic interactions, ionic interactions and hydrogen bonds.
More interactions will be added in the near future, such as aromatic-aromatic, aromatic-sulphure and cation-pi interactions.
Such interactions have a strong role in the secondary and tertiary structure of a protein, constraining helices and sheets, bonding different monomeres together, and constraining which parts of the protein chain will be exposed and thus available for inter-protein interactions (with an enzyme or a target, for example).
The number of bonds will also affect how robust against environment changes (for example, a higher number of hydrogen bonds will heighten the temperature necessary to denaturate the whole protein).
Assessing these interactions in a simple manner is therefore a crucial part of organic chemistry.

\subsection{Disulphide bonds}
Disulphide bonds are the result of a covalent bond between the sulphurs of two cystein amino-acids in a protein. They are detected by the program when the bond length is below 2.2\si{\angstrom}.

\subsection{Hydrophobic interactions}
Hydrophobic interactions happen between hydrophobic residues : Valin, Alanin, Leucin, Isoleucin, Methionine, Phenylalanin,Tryptophane, Prolin, Tyrosin.
The distance has to be below 5\si{\angstrom}. 

\subsection{Ionic interactions}
Ionic interactions happen between charged residues. The program finds couple between positive (Asparagin, Glutamin) and negative (Lysin, Arginin, Histidin) residues which distance is inferior to 6\si{\angstrom}.

\subsection{Hydrogen bonds}
Hydrogen bonds are found in 3 different scenarios : between atoms of the main chain (thus constraining the secondary structures, helix and sheets), between atoms of the aminoacids' residues, and between the main chain and residues.
\subsection{Main chain/main chain}
As of today, only hydrogen bonds between residues of the main chain can be found.
The distance between the acceptor and the donor of hydrogen must be inferior to 3.5\si{\angstrom}.

\section{Material and Methods}
\subsection{Description of the algorithm}
The script command.py requires a PDB file as input, containing at the very least the full list of atoms in the protein, as well as their coordinates.
The script acts as follows :
\begin{itemize}
	\item After input from the user, it parses the file and stocks each atom (and its characteristics) in a list.
	\item Depending on the searches the user specified, it will run a parse for each individual search, eliminating atoms or amino-acids that do not have a role in the specified interaction.
	\item The program then compares in this new list the distances between atoms. If the interaction respects the interactions's criteria and if the distance is in the range of what is expected for this type of interaction, the couple of atoms and the distance between them is stocked as a list in a list.
\end{itemize}

The results are then displayed in a CSV file.
Each section is announced by a title, then each bond appears as a line precising the position, the type (as the 3-letter code) and the chain of both residues involved in the interaction.
The distance of the interaction can be fetched for the hydrogen bonds.

\section{Results}
Multiple molecules were tested with the algorithm : 2eti and 6pa8 were used as test samples. 2eti is a small protein, enabling fast parsing and debugging, while 6pa8 is a homomeric macromolecule containing 4 chains, which presents at least one interaction of each type.

\section{Conclusion}
PIC is a promising tool providing a first glance at the interactions at the intraprotein level.

\newpage
\section{Annexes}
\subsection{Example of use}
Quick start :
\begin{center}\texttt{python3 run.py 6pa8.pdb --hydrophobic --disulfide --ionic}
\end{center}
For an optimal use of this program, please follow these steps :
\begin{itemize}
	\item Download the whole directory of the project, do not change the file tree
	\item Install MiniConda (follow this documentation : https://docs.conda.io/en/latest/miniconda.html)
	\item Import the environment from the \texttt{pic$\_$env.yml} file
	\item Activate the \texttt{pic$\_$env} environment
	\item In a terminal, in the main directory, type
	 \begin{center}\texttt{python3 run.py YOURPDB.pdb --argument}
	 \end{center}
\end{itemize}
\subsection{Future implementations}
As this project confronted the author to their first real experience with OOP, the subject was not finished in its entirety, and priority was placed in creating a functional, documented and clean base code on which the other functions could be added on a later stage.
Missing functions are : 
\begin{itemize}
	\item finding aromatic-aromatic interactions,
	\item finding hydrogen bonds between the side chains and the main chain,
	\item finding hydrogen bonds between side chains,
	\item finding aromatic-sulphur interactions,
	\item finding cation-π interactions,
	\item applying each interaction research to different monomeres of a protein,
	\item letting the user place the threshold for each interaction.
\end{itemize}

\end{document}
